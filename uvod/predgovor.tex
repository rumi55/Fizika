\chapter*{Predgovor}


	\section*{ ... }
	Treba napisati
	\section*{Kako se rješavaju zadaci?}
	Fizika predstavlja većini jedan od najstrašnijih predmeta u osnovnoj i srednjoj školi. Posebno takvom mišljenju doprinose zadaci, koje vrlo često ne znamo kako da počnemo. Ovaj tekst daće vam nekoliko glavnih smjernica za rješavanje zadataka iz fizike.
	
	Prvo i možda najvažnije pravilo jeste da se zadatak dobro pročita, s ovim dobro mislim na pažljivo i studiozno čitanje koje obično ide u tri etape. Prva etapa (ili prvo čitanje) je čisto da se informišemo  o čemu naš zadatak govori. Mnogi se plaše dugih zadataka, tj zadataka sa puno teksta, no obično su oni najlakši, jer sa puno teksta dolazi puno informacija koje nam kasnije pomažu da tak zadatak pravilno rijesimo.
	Druga etapa (čitanje) nam služi da zabilježimo sve podatke koje smo dobili u tekstu. Uobičajeno je  da se pišu jedan ispod drugoga, te da se na kraju povuče linija ispod koje pišemo šta se zapravo traži od nas da se uradi u tom zadatku. Treće čitanje (etapa) je vrlo važno, jer u njemu povezujemo sve podatke koje smo dobili.
	Kada smo završili sa iščitavanjem red je da se bacimo na rješavanje problema. Moj savjet je da se na papiru ispišu sve formule koje znamo, a da su relevantne za dati nam zadatak. Relevantne su one koje u sebi sadrže neke od podataka koje smo dobili, a pored toga moraju da odgovaraju i uslovima opisanim u problemu. Kad smo to uradili, sad je na nama da ih spojimo kao slagalicu. Da na jednoj strani imamo podatak koji se traži od nas, a na drugoj sve poznate informacije iz opisa zadatka.
	
	Ne zaboravite provjeriti cjelokupan zadatak još jednom, nakon proračuna. Provjerite rješenje i kompletan računski dio zadatka, jer se po pravilu uvijek potkrade neka sitna greška, koja nas  na kraju košta vrlo važnih bodova.
	Važno je istaći i to koliko sam rezultat ima smisla. Npr.  za rezultat ste dobili temperaturu sobe od 1000 stepeni celzijusa, ili automobil se kreće brzinom 3000 km na čas. Ovakve rezultati trebali bi da vas upozore da ste negdje pogriješili i da bi trebali još jednom da se vratite  kroz čitav zadatak.
	
	\begin{flushright}
		Autor
	\end{flushright}
	\newpage
	\pagenumbering{arabic}
	\patchcmd{\chapter}
	{\clearpage}
	{\cleardoublepage}
	{}
	{}
		\pagestyle{headings}